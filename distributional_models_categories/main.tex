\documentclass{article}

\usepackage[english]{babel}

\usepackage[letterpaper,top=2cm,bottom=2cm,left=3cm,right=3cm,marginparwidth=1.75cm]{geometry}

% Useful packages
\usepackage{amsmath}
\usepackage{graphicx}
\usepackage[colorlinks=true, allcolors=blue]{hyperref}
\newtheorem{theorem}{Theorem}[section]
\newtheorem{corollary}{Corollary}[theorem]
\newtheorem{lemma}[theorem]{Lemma}

\title{Categories of distributional language models}
\author{Dmitriy Akhmediyev}

\begin{document}
\maketitle

\begin{abstract}
Cartesian closure of so called semantic copresheaves and their enriched [0,1]-version
\end{abstract}

\section{Introduction}

Notes about categorification of LLM's, especially distributional, and 

\section{Intro}

\section{Cartesian closure as implication}

\section{Random (co)products and Cartesian closures}
\subsection{Random products}
AND "operator"
\newline
$h^{red} \times h^{rand}$
\newline
$h^{red} \times h^{rand($\leq \delta$)}$
\newline
\newline
$"semantic circle of radius $\delta$"$
\newline
as example: 
\newline
Red and blue -> list of objects with that property: "flag", "coin" ...
\newline
\newline
$"semantic circle of radius zero"$
\newline
\delta=0
\newline
$h^{red} \times h^{red}$
\newline
list of objects with "red and red" property: "red lion with red crown" ...

\subsection{Random coproducts}
OR "operator"
\subsection{Random Cartesian closures}
Implication "operator"

\section{Scope ambiguities}
- Every man loves a woman  
\newline
- Work with scope ambiguities in wider context (lemma): 
\newline
\begin{lemma}
There is exist string of every length (as objects in L category)
\end{lemma}

\bibliographystyle{alpha}
\bibliography{sample}

\end{document}
